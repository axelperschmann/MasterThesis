% -- Vektor fett darstellen -----------------
% \let\oldvec\vec
% \def\vec#1{{\boldsymbol{#1}}} %Fetter Vektor
% \newcommand{\ve}{\vec} %
% -------------------------------------------

\newcommand{\itemmath}[1]{$\bm{\mathsf{#1}}$}
\newcommand{\captionmath}[1]{$\protect\captionmathfont{#1}$}
  \newcommand{\captionmathfonttoc}[1]{\mathsf{#1}}
  \newcommand{\captionmathfonttext}[1]{\bm{\mathsf{#1}}}
  \let\captionmathfont=\captionmathfonttext
\newcommand{\sectionmath}[1]{$\protect\sectionmathfont{#1}$}
  \newcommand{\sectionmathfonttoc}[1]{#1}
  \newcommand{\sectionmathfonttext}[1]{\mathsf{#1}}
  \let\sectionmathfont=\sectionmathfonttext

%% from package braket
{\catcode`\|=\active
  \xdef\Concat{\protect\expandafter\noexpand\csname Concat \endcsname}
  \expandafter\gdef\csname Concat \endcsname#1{\begingroup
     \ifx\SavedDoubleVert\relax
       \let\SavedDoubleVert\|\let\|\BraDoubleVert
     \fi
     \mathcode`\|32768\let|\BraVert
     \left[{#1}\right]\endgroup}
}

\newcommand{\mathup}[1]{\mathrm{#1}}

\newcommand\eref[1]{\tag*{\ref{#1}}} % Formel erscheint erneut und soll gleiche Formelnummer wie beim ersten Auftreten erhalten

\newcommand\const{\mathrm{const}}
\newcommand\rp{^{-1}}
\newcommand\rps{^{-2}}
\newcommand\rpc{^{-3}}

\newcommand\transpose{^T} %{^\top}
\newcommand\rptranspose{^{-T}} %^{^{-\top}}
\newcommand\pseudoinverse{^{+}}

\newcommand\define{:=} % :=, \ensuremath{\mathrel{\stackrel{\mathrm{def}}{=}}}
\newcommand\ldefine{=:}
\newcommand{\setsep}{\, | \,}
%\newcommand\set[1]{\left{#1\right}}
%\newcommand\set*[2]{\left{ #1 \setsep #2 \right}}
%\newcommand{\set}[2][]{%
%    \ifthenelse{\equal{#1}{}}{\left\{ #2 \right\}}{\left\{ #1 \setsep #2 \right\}}%
%  }

\newcommand{\norm}[1]{\left\Vert #1 \right\Vert_2}
\newcommand{\floor}[1]{\left\lfloor #1 \right\rfloor}
\newcommand{\ceil}[1]{\left\lceil #1 \right\rceil}

%\newcommand{\vecval}[1]{\mathbf{#1}}
\newcommand{\vecval}[1]{{\color{MathsVectorColor}\bm{#1}}}
\newcommand{\matval}[1]{{\color{MathsMatrixColor}#1}}
  %\newcommand{\vecval}[1]{\bm{#1}}
  %\newcommand{\matval}[1]{#1}

\newcommand{\sspace}{\quad}
\newcommand{\wspace}{\qquad}

\newcommand{\sand}{\sspace\text{and}\sspace}
\newcommand{\wand}{\wspace\text{and}\wspace}
\newcommand{\for}{\text{for }}

\newcommand{\R}{\mathbb{R}}
\newcommand{\N}{\mathbb{N}}

\newcommand{\unitvec}[2]{{\vecval{u}_{#1}^{(#2)}}}
\newcommand{\identitymat}[1]{\matval{\mathup{I}}_{#1,#1}}
\newcommand{\permutemat}{\matval{\Pi}}
\newcommand{\zerovec}[1]{\vecval{0}_{#1}}
\newcommand{\zeromat}[2]{\matval{0}_{#1,#2}}

\newcommand{\eqnannotate}[1]{\tag*{\color{AnnotationColor} #1}}
\newcommand{\eqnannotatefrom}[2][]{\eqnannotate{#1 from \Cref{#2}}}
\newcommand{\eqnannotatecf}[1]{\eqnannotate{\cf (\explicitref{#1})}}
\newcommand{\equalref}[1]{\,\stackrel{\text{\footnotesize(\explicitref{#1})}}{=}\,}

\newcommand{\fulfill}{\stackrel{!}{=}}
\newcommand{\inlineortho}{\bot}
\newcommand{\ortho}{\,\, \inlineortho \,\,}
\newcommand{\iszero}[1]{\underbrace{#1}_{=0}}

\newcommand{\vecsize}[1]{\in\R^{#1}}
\newcommand{\matsize}[2]{\in\R^{#1 \times #2}}

\newcommand{\Tsstep}[1]{\matval{\overline{#1}}}
\newcommand{\sstep}[1]{\matval{\overline{#1}}}
\newcommand{\Sstep}[1]{\matval{\ddot{#1}}}
\newcommand{\dual}[1]{\matval{\hat{#1}}}
\newcommand{\dualvec}[1]{\hat{#1}}
\newcommand{\append}[1]{\underline{#1}}

%\newcommand{\ATpower}[1]{{\left( \matval{A}\transpose \right)}^{#1}}
\newcommand{\ATpower}[1]{{( \matval{A}\transpose )}^{#1}}
\newcommand{\bracketT}[1]{{\left( #1 \right)}\transpose}

\newcommand{\abs}[1]{\left\lvert #1 \right\rvert}
\newcommand{\linearspan}[1]{\mathrm{span}\left( #1 \right)}
\newcommand{\conj}[1]{\overline{#1}}

\newcommand{\concatmat}[1]{\Concat{#1}}
\newcommand{\concatvec}[1]{\left[#1\right]}
\newcommand{\concatmatsep}{\, | \,}
%\newcommand{\lincomb}[1]{\left\langle #1 \right\rangle}
\newcommand{\lincomb}[1]{\linearspan{#1}}
%\newcommand{\vecprod}[2]{\left( #1, #2 \right)}
\newcommand{\vecprod}[2]{\left\langle #1, #2 \right\rangle}

%\newcommand{\sign}[1]{\mathrm{sign}\left( #1 \right)}
%\newcommand{\ssign}{\mathrm{sign}}
\newcommand{\diag}{\operatorname{diag}}
%\renewcommand\Re[1][]{\mathrm{Re}\,#1}
%\renewcommand\Im[1][]{\mathrm{Im}\,#1}

\newcommand{\vvector}[2]{\left[\begin{array}{c} #1 \\ #2 \end{array} \right]}
\newcommand{\vvvector}[3]{\left[\begin{array}{c} #1 \\ #2 \\ #3 \end{array} \right]}

\newcommand{\intervaloo}[2]{\left(#1,#2\right)}
\newcommand{\intervaloc}[2]{\left(#1,#2\right]}
\newcommand{\intervalco}[2]{\left[#1,#2\right)}
\newcommand{\intervalcc}[2]{\left[#1,#2\right]}

\newenvironment{mmatrix}{\begin{bmatrix}}{\end{bmatrix}}

\newcommand{\Oh}[1]{\mathrm{O}\left(#1\right)}

\newcommand{\e}[1]{\mathup{e}^{#1}}

\newcommand{\dd}{\mathop{}\!\mathrm{d}}
\newcommand{\Laplace}{\mathrm{\Delta}}

\newcommand\op[1]{{\hat{\mathrm{#1}}}}  % Operator

\newcommand\imaginary{\mathup{i}}

% use together with long limits in \sum, \prod, etc
% from mathmode, p. 63
  \def\clap#1{\hbox to 0pt{\hss#1\hss}}
  \def\mathclap{\mathpalette\mathclapinternal}
  \def\mathclapinternal#1#2{%
    \clap{$\mathsurround=0pt#1{#2}$}
  }
	
	
\newcommand{\argmax}{\operatornamewithlimits{arg \, max}}

