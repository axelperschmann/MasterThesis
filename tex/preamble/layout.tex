% ~~~~~~~~~~~~~~~~~~~~~~~~~~~~~~~~~~~~~~~~~~~~~~~~~~~~~~~~~~~~~~~~~~~~~~~~
% layout packages
% ~~~~~~~~~~~~~~~~~~~~~~~~~~~~~~~~~~~~~~~~~~~~~~~~~~~~~~~~~~~~~~~~~~~~~~~~

%%% Diverse Pakete und Einstellungen =====================================
 
%%% Doc: Documentation inside dtx file
% Mehrere Text-Spalten
\usepackage{multicol}

\usepackage{lscape}
\usepackage{rotating}
 
%\nonfrenchspacing     % liefert extra Platz hinter Satzenden.
                                % Fuer deutschen Text standardmaessig ausgeschaltet!
 
\usepackage{ellipsis}  % >>Intelligente<< \dots

%% bei Zweiseitigem Druck besser auskommentieren
\clubpenalty = 10000        % Disable single lines at the start of a paragraph (Schusterjungen)
\widowpenalty = 10000       % Disable single lines at the end of a paragraph (Hurenkinder)
\displaywidowpenalty = 10000

%% Zeilenabstand =========================================================
%
%%% Doc: ftp://tug.ctan.org/pub/tex-archive/macros/latex/contrib/setspace/setspace.sty
\usepackage{setspace}
%\doublespace          % 2-facher Abstand
\onehalfspace        % 1,5-facher Abstand
% hereafter load 'typearea' again

%% Seitenlayout ==========================================================
%
% Layout laden um im Dokument den Befehl \layout nutzen zu koennen
%%% Doc: no documentation
%\usepackage[verbose]{layout}
%
 
% Layout mit 'geometry'
%%% Doc: ftp://tug.ctan.org/pub/tex-archive/macros/latex/contrib/geometry/manual.pdf
\usepackage{geometry}
 
\IfPackageLoaded{geometry}{%
\geometry{%
%%% Paper Groesse
   a4paper, % Andere a0paper, a1paper, a2paper, a3paper, , a5paper, a6paper,
            % b0paper, b1paper, b2paper, b3paper, b4paper, b5paper, b6paper
            % letterpaper, executivepaper, legalpaper
   %screen,  % a special paper size with (W,H) = (225mm,180mm)
   %paperwidth=,
   %paperheight=,
   %papersize=, %{ width , height }
   %landscape,  % Querformat
   portrait,    % Hochformat
%%% Koerper Groesse
   %hscale=,      % ratio of width of total body to \paperwidth
                  % hscale=0.8 is equivalent to width=0.8\paperwidth. (0.7 by default)
   %vscale=,      % ratio of height of total body to \paperheight
                  % vscale=0.9 is equivalent to height=0.9\paperheight.
   %scale=,       % ratio of total body to the paper. scale={ h-scale , v-scale }
   %totalwidth=,    % width of total body % (Generally, width >= textwidth)
   %totalheight=,   % height of total body, excluding header and footer by default
   %total=,        % total={ width , height }
   %textwidth=,    % modifies \textwidth, the width of body
   %textheight=,   % modifies \textheight, the height of body
   %body=,        % { width , height } sets both \textwidth and \textheight of the body of page.
 %  lines=50,       % enables users to specify \textheight by the number of lines.
   %includefoot,  % includes the foot of the page, \footskip, into body.
   %includehead,  % includes the head of the page, \headheight and \headsep, into total body.
   %includeheadfoot, % sets both includehead and includefoot to true
   %includemp,    % includes the margin notes, \marginparwidth and \marginparsep, into body
   %includeall,   % sets both includeheadfoot and includemp to true.
   %ignorehead,   % disregards the head of the page, headheight and headsep in determining vertical layout
   %ignorefoot,   % disregards the foot of page, footskip, in determining vertical layout
   %ignoreheadfoot, % sets both ignorehead and ignorefoot to true.
   %ignoremp,     % disregards the marginal notes in determining the horizontal margins
   ignoreall,     % sets both ignoreheadfoot and ignoremp to true
   heightrounded, % This option rounds \textheight to n-times (n: an integer) of \baselineskip
   %hdivide=,     % { left margin , width , right margin }
                  % Note that you should not specify all of the three parameters
   %vdivide=,     % { top margin , height , bottom margin }
   %divide=,      % ={A,B,C} %  is interpreted as hdivide={A,B,C} and vdivide={A,B,C}.
 %Margin
   lmargin=2.5cm,        % left margin (for oneside) or inner margin (for twoside) of total body
                  % alias: lmargin, inner
   rmargin=2.5cm,       % right or outer margin of total body
                  % alias: rmargin outer
   tmargin=2.5cm,         % top margin of the page.
                  % Alias : tmargin
   bmargin=2cm,      % bottom margin of the page
                  % Alias : bmargin
   %hmargin=,     % left and right margin. hmargin={ left margin , right margin }
   %vmargin=,     % top and bottom margin. vmargin={ top margin , bottom margin }
   %margin=,      % margin={A,B} is equivalent to hmargin={A,B} and vmargin={A,B}
   %hmarginratio, % horizontal margin ratio of left (inner) to right (outer).
   %vmarginratio, % vertical margin ratio of top to bottom.
   %marginratio,  % marginratio={ horizontal ratio , vertical ratio }
   %hcentering,   % sets auto-centering horizontally and is equivalent to hmarginratio=1:1
   %vcentering,   % sets auto-centering vertically and is equivalent to vmarginratio=1:1
   %centering,    % sets auto-centering and is equivalent to marginratio=1:1
  % twoside,       % switches on twoside mode with left and right margins swapped on verso pages.
   %asymmetric,   % implements a twosided layout in which margins are not swapped on alternate pages
                  % and in which the marginal notes stay always on the same side.
   %bindingoffset=5mm,  % removes a specified space for binding
%%% Dimensionen
   %headheight=,  % Alias:  head
   %headsep=,     % separation between header and text
   %footskip=,    % distance separation between baseline of last line of text and baseline of footer
                  % Alias: foot
   %nohead,       % eliminates spaces for the head of the page
                  % equivalent to both \headheight=0pt and \headsep=0pt.
   %nofoot,       % eliminates spaces for the foot of the page
                  % equivalent to \footskip=0pt.
   %noheadfoot,   % equivalent to nohead and nofoot.
   %footnotesep=, % changes the dimension \skip\footins,.
                  % separation between the bottom of text body and the top of footnote text
  % marginparwidth=0pt, % width of the marginal notes
                  % Alias: marginpar
   %marginparsep=,% separation between body and marginal notes.
   %nomarginpar,  % shrinks spaces for marginal notes to 0pt
   %columnsep=,   % the separation between two columns in twocolumn mode.
   %hoffset=0.6cm,
   %voffset=,
   %offset=,      % horizontal and vertical offset.
                  % offset={ hoffset , voffset }
   %twocolumn,    % twocolumn=false denotes onecolumn
   %twoside,
   %textwidth=380pt,   % sets \textwidth directly
   %textheight=,  % sets \textheight directly
   %reversemp,    % makes the marginal notes appear in the left (inner) margin 
                  % Alias: reversemarginpar
}
} % Endif
 
% - Anzeigen des Layouts -
\IfPackageLoaded{geometry}{%
%   \geometry{showframe}
}
 
% Layout mit 'typearea'
%%% Doc: ftp://tug.ctan.org/pub/tex-archive/macros/latex/contrib/koma-script/scrguide.pdf
\usepackage{typearea} 
 
\IfPackageLoaded{typearea}{% Wenn typearea geladen ist
   \IfPackageNotLoaded{geometry}{% aber nicht geometry
      \typearea[current]{last}
   }
}
 
% BCOR
%    current  % Satzspiegelberechnung mit dem aktuell g�ltigen BCOR-Wert erneut
%             % durchf�hren.
% DIV
%    calc     % Satzspiegelberechnung einschlie�lich Ermittlung eines guten
%             % DIV-Wertes erneut durchf�hren.
%    classic  % Satzspiegelberechnung nach dem
%             % mittelalterlichen Buchseitenkanon
%             % (Kreisberechnung) erneut durchf�hren.
%    current  % Satzspiegelberechnung mit dem aktuell g�ltigen DIV-Wert erneut
%             % durchf�hren.
%    default  % Satzspiegelberechnung mit dem Standardwert f�r das aktuelle
%             % Seitenformat und die aktuelle Schriftgr��e erneut durchf�hren.
%             % Falls kein Standardwert existiert calc anwenden.
%    last     % Satzspiegelberechnung mit demselben DIV -Argument, das beim
%             % letzten Aufruf angegeben wurde, erneut durchf�hren
 
 
%\usepackage[colorgrid,texcoord,gridunit=mm]{showframe}
 
\raggedbottom     % Variable Seitenhoehen zulassen

%% Aussehen der URLS======================================================
 
%fuer URL (nur wenn url geladen ist)
\IfDefined{urlstyle}{
  \urlstyle{tt} %sf
}

%% Kopf und Fusszeilen====================================================
%%% Doc: ftp://tug.ctan.org/pub/tex-archive/macros/latex/contrib/koma-script/scrguide.pdf
\usepackage[%
   automark,         % automatische Aktualisierung der Kolumnentitel
   nouppercase,      % Grossbuchstaben verhindern
   %markuppercase    % Grossbuchstaben erzwingen
   %markusedcase     % vordefinierten Stil beibehalten
   %komastyle,       % Stil von Koma Script
   %standardstyle,   % Stil der Standardklassen
]{scrpage2}
 
\IfElseChapterDefined{%
   \pagestyle{scrheadings} % Seite mit Headern
}{
   \pagestyle{scrplain} % Seiten ohne Header
}
%\pagestyle{empty} % Seiten ohne Header
%
% loescht voreingestellte Stile
\clearscrheadings
\clearscrplain
%
% Was steht wo...
\IfElseChapterDefined{
   % Oben aussen: Kapitel und Section
   % Unten aussen: Seitenzahl
   % \ohead{\headmark} % Oben au�en: Setzt Kapitel und Section automatisch
   % \ofoot[\pagemark]{\pagemark}
   % oder...
   % Oben aussen: Seitenzahlen
   % Oben innen: Kapitel und Section
	 \ohead{\pagemark}
   \ihead{\headmark}
}{
   %\cfoot[\pagemark]{\pagemark} % Mitte unten: Seitenzahlen bei plain
}


\ohead[\pagemark]{\pagemark}
\ihead[\headmark]{\headmark}
% Vollstaendige Liste der moeglichen Positionierungen
% \lehead[scrplain-links-gerade]{scrheadings-links-gerade}
% \cehead[scrplain-mittig-gerade]{scrheadings-mittig-gerade}
% \rehead[scrplain-rechts-gerade]{scrheadings-rechts-gerade}
% \lefoot[scrplain-links-gerade]{scrheadings-links-gerade}
% \cefoot[scrplain-mittig-gerade]{scrheadings-mittig-gerade}
% \refoot[scrplain-rechts-gerade]{scrheadings-rechts-gerade}
% \lohead[scrplain-links-ungerade]{scrheadings-links-ungerade}
% \cohead[scrplain-mittig-ungerade]{scrheadings-mittig-ungerade}
% \rohead[scrplain-rechts-ungerade]{scrheadings-rechts-ungerade}
% \lofoot[scrplain-links-ungerade]{scrheadings-links-ungerade}
% \cofoot[scrplain-mittig-ungerade]{scrheadings-mittig-ungerade}
% \rofoot[scrplain-rechts-ungerade]{scrheadings-rechts-ungerade}
% \ihead[scrplain-innen]{scrheadings-innen}
% \chead[scrplain-zentriert]{scrheadings-zentriert}
% \ohead[scrplain-au�en]{scrheadings-au�en}
% \ifoot[scrplain-innen]{scrheadings-innen}
% \cfoot[scrplain-zentriert]{scrheadings-zentriert}
% \ofoot[scrplain-au�en]{scrheadings-au�en}
 
 
%\usepackage{lastpage} % Stellt 'LastPage' zur Verfuegung
%\cfoot[Seite \pagemark~von \pageref{LastPage}]{} % Seitenzahl von Anzahl Seiten
 
% Angezeigte Abschnitte im Header
\IfElseChapterDefined{
   \automark[section]{chapter} %[rechts]{links}
}{
   \automark[section]{section} %[rechts]{links}  <- Chapter im header (z.b. bei k�rzerem scrartcl)       
	  %\automark[chapter]{section} %[rechts]{links} <- Section im header
}
%
% Linien (moegliche Kombination mit Breiten)
%\IfChapterDefined{
   %\setheadtopline{}     % modifiziert die Parameter fuer die Linie ueber dem Seitenkopf
   \setheadsepline{.4pt}[\color{black}]
                         % modifiziert die Parameter fuer die Linie zwischen Kopf
                         % und Textk�rper
   %\setfootsepline{}    % modifiziert die Parameter fuer die Linie zwischen Text
                         % und Fu�
   %\setfootbotline{}    % modifiziert die Parameter fuer die Linie unter dem Seitenfuss
%}

% Groesse des Headers
\setlength{\headheight}{1.1\baselineskip}
% -> eingestellt �ber Option 'headlines'.
 
% Breite von Kopf und Fusszeile einstellen
% \setheadwidth[Verschiebung]{Breite}
% \setfootwidth[Verschiebung]{Breite}
% m�gliche Werte
% paper - die Breite des Papiers
% page - die Breite der Seite
% text - die Breite des Textbereichs
% textwithmarginpar - die Breite des Textbereichs inklusive dem Seitenrand
% head - die aktuelle Breite des Seitenkopfes
% foot - die aktuelle Breite des Seitenfusses
\setheadwidth[0pt]{text}
\setfootwidth[0pt]{text}

%% Fussnoten =============================================================
% Keine hochgestellten Ziffern in der Fussnote (KOMA-Script-spezifisch):
\deffootnote{1.5em}{1em}{\makebox[1.5em][l]{\thefootnotemark}}
\addtolength{\skip\footins}{\baselineskip} % Abstand Text <-> Fussnote
 
\setlength{\dimen\footins}{10\baselineskip} % Beschraenkt den Platz von Fussnoten auf 10 Zeilen
 
\interfootnotelinepenalty=10000 % Verhindert das Fortsetzen von
                                % Fussnoten auf der gegen�berligenden Seite
                                
%% Schriften (Sections )==================================================

\IfElsePackageLoaded{fourier}{
   \newcommand\SectionFontStyle{\rmfamily}
}{
   \newcommand\SectionFontStyle{\sffamily} %alt. \sffamily
}

% -- Koma Schriften --
\IfChapterDefined{%
   \setkomafont{chapter}{\SectionFontStyle}    % Chapter
}

\setkomafont{sectioning}{\SectionFontStyle} %  % Titelzeilen % \bfseries
\setkomafont{pagenumber}{\small\SectionFontStyle}             % Seitenzahl
\setkomafont{pageheadfoot}{\small\sffamily}        % Kopfzeile
%\setkomafont{pagefoot}{\small\sffamily}        % Kopfzeile
\setkomafont{descriptionlabel}{\itshape}        % Kopfzeile
%
\addtokomafont{minisec}{\bfseries}  %Bold minisec
\addtokomafont{sectioning}{\color{SectionColor}} % Farbe der Ueberschriften
\IfChapterDefined{%
  \addtokomafont{chapter}{\color{SectionColor}} % Farbe der Ueberschriften
}
\renewcommand*{\raggedsection}{\raggedright} % Titelzeile linksbuendig, haengend

%% UeberSchriften (Chapter und Sections) =================================
% -- Ueberschriften komlett Umdefinieren --
%%% Doc: ftp://tug.ctan.org/pub/tex-archive/macros/latex/contrib/titlesec/titlesec.pdf
\usepackage{titlesec}
 
% -- Section Aussehen veraendern --
% --------------------------------
%% -> Section mit Unterstrich
% \titleformat{\section}
%   [hang]%[frame]display
%   {\usekomafont{sectioning}\Large}
%  {\thesection}
%   {6pt}
%   {}
%   [\titlerule \vspace{0.5\baselineskip}]
% --------------------------------
 
% -- Chapter Aussehen veraendern --
% --------------------------------
%--> Box mit (Kapitel + Nummer ) +  Name
% \titleformat{\chapter}[display]     % {command}[shape]
%   {\usekomafont{chapter}\filcenter} % format
%   {                                 % label
%   {\fcolorbox{black}{shadecolor}{
%   {\huge\chaptertitlename\mbox{\hspace{1mm}}\thechapter}
%   }}}
%   {1pc}                             % sep (from chapternumber)
%   {\vspace{1pc}}                    % {before}[after] (before chaptertitle and after)
% --------------------------------
%--> Kapitel + Nummer + Trennlinie + Name + Trennlinie
%\titleformat{\chapter}[display]  % {command}[shape]
%  {\usekomafont{chapter}\Large \color{black}}  % format
%  {                       % label
%  \LARGE\MakeUppercase{\chaptertitlename} \Huge \thechapter \filright%
%  }%}
%  {1pt}                    % sep (from chapternumber)
%  {\titlerule \vspace{0.9pc} \filright \color{SectionColor}}   % {before}[after] (before chaptertitle and after)
%  [\color{black} \vspace{0.9pc} \filright {\titlerule}]
% --------------------------------
%--> Nummer + Name


%--> �berschriften definieren!
\titleformat{\chapter}[hang]{\usekomafont{sectioning}\Large\rmfamily\bfseries \color{black}}{\thechapter\filright}{1pc}{\phantomsection}
\titleformat{\section}[hang]{\usekomafont{sectioning}\Large\rmfamily\bfseries \color{black}}{\thesection\filright}{1pc}{\phantomsection} 
\titleformat{\subsection}[hang]{\usekomafont{sectioning}\Large\rmfamily\bfseries \color{black}}{\thesubsection\filright}{1pc}{\phantomsection} 
\titleformat{\subsubsection}[hang]{\usekomafont{sectioning}\large\rmfamily\bfseries \color{black}}{\thesubsubsection\filright}{1pc}{\phantomsection} 

%%% Doc: No documentation
% Indent first paragraph after section header
% \usepackage{indentfirst}

%% Add line break after \paragraph{}
  \let\ORGparagraph=\paragraph
  \newcommand{\MYparagraph}[1]{\hfill\vspace{0.1cm}\parbox{\linewidth}{\ORGparagraph{#1}\hfill\vspace{0.1cm}}\reallynopagebreak}
  \let\paragraph=\MYparagraph

%% Captions (Schrift, Aussehen) ==========================================
 
% % Folgende Befehle werden durch das Paket caption und subfig ersetzt !
% \setcapindent{1em} % Einrueckung der Beschriftung
%\setkomafont{caption}{\color{black}\small\sffamily\RaggedRight}  % Schrift fuer Caption
% \setkomafont{captionlabel}{\color{black}\small}   % Schrift fuer 'Abbildung' usw.
 
%%% Doc: ftp://tug.ctan.org/pub/tex-archive/macros/latex/contrib/caption/caption.pdf
\usepackage{caption}
% Aussehen der Captions
\captionsetup{
   margin = 10pt,
   font = {rm},
   labelfont = {bf},
   format = plain, % oder 'hang'
   indention = 0em,  % Einruecken der Beschriftung
   labelsep = colon, %period, space, quad, newline
   justification = RaggedRight, % justified, centering
   singlelinecheck = true, % false (true=bei einer Zeile immer zentrieren)
   position = bottom %top
}
%%% Bugfix Workaround
\DeclareCaptionOption{parskip}[]{}
\DeclareCaptionOption{parindent}[]{}
 
% Aussehen der Captions fuer subfigures (subfig-Paket)
\IfPackageLoaded{subfig}{
 \captionsetup[subfloat]{%
   margin = 10pt,
   font = {small,rm},
   labelfont = {small,bf},
   format = plain, % oder 'hang'
   indention = 0em,  % Einruecken der Beschriftung
   labelsep = space, %period, space, quad, newline
   justification = RaggedRight, % justified, centering
   singlelinecheck = true, % false (true=bei einer Zeile immer zentrieren)
   position = bottom, %top
   labelformat = parens % simple, empty % Wie die Bezeichnung gesetzt wird
 }
}
 
% Aendern der Bezeichnung fuer Abbildung und Tabelle
% \addto\captionsngerman{% "captionsgerman" fuer alte  Rechschreibung
%   \renewcommand{\figurename}{Abb.}%
%   \renewcommand{\tablename}{Tab.}%
% }
 
% Caption fuer nicht fliessende Umgebungen
%%% Doc: ftp://tug.ctan.org/pub/tex-archive/macros/latex/contrib/misc/capt-of.sty
\IfPackageNotLoaded{caption}{
  \usepackage{capt-of} % only load when caption is not loaded. Otherwise compiling will fail.
  %Usage: \captionof{table}[short Titel]{long Titel}
}
%
 
 
%%% Doc: ftp://tug.ctan.org/pub/tex-archive/macros/latex/contrib/mcaption/mcaption.pdf
% Captions in Margins
% \usepackage[
%   top,
%   bottom
% ]{mcaption}
 
%%% Example:
% \begin{figure}
%   \begin{margincap}[short caption]{margin caption}
%     \centering
%     \includegraphics{picture}
%   \end{margincap}
% \end{figure}
 
 
 
% \numberwithin{figure}{chapter} %Befehl zum Kapitelweise Nummerieren der Bilder, setzt `amsmath' vorraus
% \numberwithin{table}{chapter}  %Befehl zum Kapitelweise Nummerieren der Tabellen, setzt `amsmath' vorraus

%% Inhaltsverzeichnis (Schrift, Aussehen) sowie weitere Verzeichnisse ====
 
% http://mirror.ctan.org/macros/latex/contrib/tocvsec2/tocvsec2.pdf
% Change the tocdepth within document, e.g. adjust for appendix
\usepackage{tocvsec2}
  \maxtocdepth{section}
 
\setcounter{secnumdepth}{3}    % Abbildungsnummerierung mit groesserer Tiefe
\setcounter{tocdepth}{2}     % Inhaltsverzeichnis mit groesserer Tiefe
%
 
% Inhalte von List of Figures
\IfPackageLoaded{subfig}{
  \setcounter{lofdepth}{1}  %1 = nur figures, 2 = figures + subfigures
}
 
% -------------------------------------------------------
 
% Aussehen des Inhaltsverzeichnisses: tocloft
%%% Doc: ftp://tug.ctan.org/pub/tex-archive/macros/latex/contrib/tocloft/tocloft.pdf
%% Laden mit Option subfigure in Abhaengigkeit vom Paket subfigure und subfig
% \IfElsePackageLoaded{subfig}
%   % IF subfig
%   {\usepackage[subfigure]{tocloft}}{
%   % ELSE
%   \IfElsePackageLoaded{subfigure}
%     % IF subfigure
%     {\usepackage[subfigure]{tocloft}}
%      % Else (No subfig nor subfigure)
%     {\usepackage{tocloft}}
%   }
%
% %TOCLOFT zerstoert Layout der Ueberschriften von TOC, LOT, LOF
% \IfPackageLoaded{tocloft}{
% %
% %%%% Layout Matthias Pospiech (alles serifenlos)
% \IfChapterDefined{%
%   \renewcommand{\cftchappagefont}{\bfseries\sffamily}  % Kapitel Seiten Schrift
%   \renewcommand{\cftchapfont}{\bfseries\sffamily}      % Kapitel Schrift
% }
% \renewcommand{\cftsecpagefont}{\sffamily}            % Section Seiten Schrift
% \renewcommand{\cftsubsecpagefont}{\sffamily}         % Subsectin Seiten Schrift
% \renewcommand{\cftsecfont}{\sffamily\bfseries}                % Section Schrift
% \renewcommand{\cftsubsecfont}{\sffamily}             % Subsection Schrift
%
% %%%% Layout aus Typokurz:
% % % Seitenzahlen direkt hinter TOC-Eintrag:
% % % Ebene \chapter
% % \renewcommand{\cftchapleader}{}
% % \renewcommand{\cftchapafterpnum}{\cftparfillskip}
% % % Ebene \section
% % \renewcommand{\cftsecleader}{}
% % \renewcommand{\cftsecafterpnum}{\cftparfillskip}
% % % Ebene \subsection
% % \renewcommand{\cftsubsecleader}{}
% % \renewcommand{\cftsubsecafterpnum}{\cftparfillskip}
% % % Abstaende vor Eintraegen im TOC verkleinern
% % \setlength{\cftbeforesecskip}{.4\baselineskip}
% % \setlength{\cftbeforesubsecskip}{.1\baselineskip}
% }
% % Ende tocloft Einstellungen --------------
 
%%% Doc: ftp://tug.ctan.org/pub/tex-archive/macros/latex/contrib/ms/multitoc.dvi
% TOC in mehreren Spalten setzen
%\usepackage[toc]{multitoc}
 
% -------------------------


%% Fix for minitoc to disable warnings related to packages titlesec, hyperref
\RequirePackage{mtcmess}[2006/03/14]
  \let\mtcPackageWarningNoLineOLD=\mtcPackageWarningNoLine
  \renewcommand{\mtcPackageWarningNoLine}[3][]{%
    \ifthenelse{\equal{#1}{W0099}}{}{ % titlesec
    \ifthenelse{\equal{#1}{W0030}}{}{ % \part altered due to hyperref
    \ifthenelse{\equal{#1}{W0023}}{}{ % \part, \chapter change due to hyperref
    \ifthenelse{\equal{#1}{W0028}}{}{ % \chapter altered due to hyperref
    \ifthenelse{\equal{#1}{W0024}}{}{ % see log for details
      \mtcPackageWarningNoLineOLD[#1]{#2}{#3}
    }}}}}
  }
 
%Schriften fuer Minitoc (Inhaltsverzeichnis vor jedem Kapitel)
%% Doc: ftp://tug.ctan.org/pub/tex-archive/macros/latex/contrib/minitoc/minitoc.pdf
%% FAQ: http://www.tex.ac.uk/tex-archive/macros/latex/contrib/minitoc/minitoc.bug
\IfElseChapterDefined{%
 \usepackage{minitoc}
 % \usepackage{mtcoff} %% disables minitoc commands
 
 \setlength{\mtcindent}{0em} % default: 24pt
 \setcounter{minitocdepth}{1}
 \setlength{\mtcskipamount}{\bigskipamount} % default: \bigskipamount
 \mtcsettitlefont{minitoc}{\normalsize\SectionFontStyle}
 \mtcsetfont{minitoc}{*}{\small\SectionFontStyle} %\color{textcolor}
 \mtcsetfont{minitoc}{section}{\small\SectionFontStyle}
 \mtcsetfont{minitoc}{subsection}{\small\SectionFontStyle}
 \mtcsetfont{minitoc}{subsubsection}{\small\SectionFontStyle} 
 %\mtcselectlanguage{german}
 \renewcommand{\beforeminitoc}{\vspace*{0cm}}
 \renewcommand{\afterminitoc}{\vspace*{1cm}\noindent} % Added \noindent as a fix
}{
 \usepackage{minitoc}
 % \usepackage{mtcoff} %% disables minitoc commands
 
 \setlength{\stcindent}{0pt} %default
 \setcounter{secttocdepth}{2} %default
 \mtcsettitlefont{secttoc}{\SectionFontStyle}
 \mtcsetfont{secttoc}{*}{\small\SectionFontStyle}%
 \mtcsetfont{secttoc}{subsection}{\small\SectionFontStyle}
 \mtcsetfont{secttoc}{subsubsection}{\small\SectionFontStyle}
}

\let\mtcPackageWarningNoLine=\mtcPackageWarningNoLineOLD
 
% Packages that MUST be loaded before minitoc !
% hyperref, caption, sectsty, varsects, fncychap, hangcaption, quotchap, romannum, sfheaders, alnumsec, captcont

\usepackage{subcaption}