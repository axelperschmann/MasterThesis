% ~~~~~~~~~~~~~~~~~~~~~~~~~~~~~~~~~~~~~~~~~~~~~~~~~~~~~~~~~~~~~~~~~~~~~~~~
% Einige Pakete muessen unbedingt vor allen anderen geladen werden
% ~~~~~~~~~~~~~~~~~~~~~~~~~~~~~~~~~~~~~~~~~~~~~~~~~~~~~~~~~~~~~~~~~~~~~~~~
 
%%% Doc: www.cs.brown.edu/system/software/latex/doc/calc.pdf
% Calculation with LaTeX
\usepackage{calc}
 
%%% Doc: ftp://tug.ctan.org/pub/tex-archive/macros/latex/required/babel/babel.pdf
% Languagesetting
\usepackage[
%  german,
%  ngerman,
  english,
%  frensh,
]{babel}

%%% Info: http://www.mail-archive.com/ctan-ann@dante.de/msg01512.html
% For adding/using translations like "Theorem/Satz" inside babel, especially for beamer
\usepackage{translator}
 
%%% Doc: ftp://tug.ctan.org/pub/tex-archive/macros/latex/contrib/xcolor/xcolor.pdf
% Farben
% Incompatible: Do not load when using pstricks !
\usepackage[
  dvipsnames,
  table % Load for using rowcolors command in tables
]{xcolor}
 
 
%%% Doc: ftp://tug.ctan.org/pub/tex-archive/macros/latex/required/graphics/grfguide.pdf
% Bilder
\usepackage[%
  %final,
  %draft % do not include images (faster)
]{graphicx}
 
%%% Doc: ftp://tug.ctan.org/pub/tex-archive/macros/latex/contrib/oberdiek/epstopdf.pdf
%% If an eps image is detected, epstopdf is automatically called to convert it to pdf format.
%% Requires: graphicx loaded
\usepackage{epstopdf}
 
 
%%% Doc: ftp://tug.ctan.org/pub/tex-archive/macros/latex/required/amslatex/math/amsldoc.pdf
% Amsmath - Mathematik Basispaket
%
% fuer pst-pdf displaymath Modus vor pst-pdf benoetigt.
\usepackage[
   centertags, % (default) center tags vertically
   %tbtags,    % 'Top-or-bottom tags': For a split equation, place equation numbers level
               % with the last (resp. first) line, if numbers are on the right (resp. left).
   sumlimits,  %(default) Place the subscripts and superscripts of summation
               % symbols above and below
   %nosumlimits, % Always place the subscripts and superscripts of summation-type
               % symbols to the side, even in displayed equations.
   intlimits,  % Like sumlimits, but for integral symbols.
   %nointlimits, % (default) Opposite of intlimits.
   namelimits, % (default) Like sumlimits, but for certain 'operator names' such as
               % det, inf, lim, max, min, that traditionally have subscripts placed underneath
               % when they occur in a displayed equation.
   %nonamelimits, % Opposite of namelimits.
   %leqno,     % Place equation numbers on the left.
   %reqno,     % Place equation numbers on the right.
   %fleqn,     % Position equations at a fixed indent from the left margin rather than
               % centered in the text column.
]{amsmath} %
% eqnarray nicht zusammen mit amsmath benutzen, siehe l2tabu.pdf f�r Hintergruende.
 
%%% Doc: http://www.ctan.org/tex-archive/macros/latex/contrib/pst-pdf/pst-pdf-DE.pdf
% Used to automatically integrate eps graphics in an pdf document using pdflatex.
% Requires ps4pdf macro !!!
% Download macro from http://www.ctan.org/tex-archive/macros/latex/contrib/pst-pdf/scripts/
%
\usepackage[%
   %active,       % Aktiviert den Extraktionsmodus (DVI-Ausgabe). Die explizite Angabe ist
                  % normalerweise unn�tig (Standard im LATEX-Modus).
   %inactive,     % Das Paket wird deaktiviert, Zu�tzlich werden die Pakete pstricks und
                  % graphicx geladen
   nopstricks,    % Das Paket pstricks wird nicht geladen.
   %draft,        % Im pdfLATEX-Modus werden aus der Containerdatei eingef�gte Grafiken nur
                  % als Rahmen dargestellt.
   %final,        % Im pdfLATEX-Modus werden aus der Containerdatei eingef�gte Grafiken
                  % vollst�ndig dargestellt (Standard).
   %tightpage,    % Die Abmessung Grafiken in der Containerdatei entsprechen denen der
                  % zugeh�rigen TEX-Boxen (Standard).
   %notightpage,  % die Grafiken in der Containerdatei nehmen
                  % mindestens die Gr��e des gesamten Blattes einnehmen.
   displaymath,   %  Es werden zus�tzlich die mathematischen Umgebungen displaymath,
                  % eqnarray und $$ extrahiert und im pdf-Modus als Grafik eingef�gt.
]{pst-pdf}
%
% Notwendiger Bugfix f�r natbib Paket bei Benutzung von pst-pdf (Version <= v1.1o)
\IfPackageLoaded{pst-pdf}{
   \providecommand\makeindex{}
   \providecommand\makeglossary{}
}{}
 
 
%% Doc: ftp://tug.ctan.org/pub/tex-archive/graphics/pstricks/README
% load before graphicx
% \usepackage{pstricks}
% \usepackage{pst-plot, pst-node, pst-coil, pst-eps}
 
% This package implements a workaround for the LaTeX bug that marginpars
% sometimes appear on the wrong margin.
% \usepackage{mparhack}
% in some case this causes an error in the index together with package pdfpages
% the reason is unkown. Therefore I recommend to use the margins of marginnote
  
%% Doc: (inside relsize.sty )
%% ftp://tug.ctan.org/pub/tex-archive/macros/latex/contrib/misc/relsize.sty
%  Set the font size relative to the current font size
\usepackage{relsize}
 
%% Doc: ftp://tug.ctan.org/pub/tex-archive/macros/latex/contrib/ms/ragged2e.pdf
% Besserer Flatternsatz (Linksbuendig, statt Blocksatz)
\usepackage{ragged2e}