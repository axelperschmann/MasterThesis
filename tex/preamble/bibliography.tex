\usepackage[                 % !! what about the ``biblatex.cfg''?
        backend=bibtex,      % (bibtex), bibtex8, biber
%        %
        style=numeric-comp,
%        bibstyle=,          % should not be used without citestyle and vice versa
%        citestyle=,
%        natbib=true,
        %
%        sorting=nty,        % (nty), nyt, nyvt, anty, anyt, anyvt, debug, none
%        sortlos=los,        % bib, (los)
%        sortcites=false,    % false
%        maxnames=3,         % <integer> (3)
%        minnames=1,         % (1)
%        maxitems=3,         % (3)
%        minitems=1,         % (1)
%        autocite=,          % inline, footnote, superscript, ...
%        autopunct=true,     % true
%        babel=none,         % (none), hyphen, other, other*
%        block=none,         % (none), space, par, nbpar, ragged
        hyperref=true,      % false
%        backref=false,      % false
%        indexing=false,     % true, (false), cite, bib
%        refsection=none,    % (none), part, chapter, section, subsection
%        refsegment=none,    % (none), part, chapter, section, subsection
%        citereset=none,     % (none), part, chapter, section, subsection
%        abbreviate=true,    % true
%        date=long,          % short, (long)
%        urldate=short,      % (short), long
%        defernums=false,    % false
%        punctfont=false,    % false
        %
%        mincrossrefs=2,     % 2
        bibencoding=inputenc,   % (ascii), inputenc, <encoding>
        %%
%        keywsort=false,     % false
    %
%     useauthor=false,   % true
%     useeditor=false,   % true
%        useprefix=true,     % false
    %
%        pagetracker=true,   % true, (false), page, spread
%     citetracker=true,   % true, (false), context, strict, constrict
%     ibidtracker=true,   % true, (false), context, strict, constrict
%     opcittracker=true,   % true, (false), context, strict, constrict
%     loccittracker=true, % true, (false), context, strict, constrict
%        terseinits=true,    % false
%     labelalpha=true,   % false
%     labelnumber=true,   % false
%     labelyear=true,   % false
%     singletitle=true,   % false
%     uniquename=true,   % true, (false), init
%
        doi=false,
        url=false,
        firstinits=true, % render first/middle names as initals
            ]{biblatex}
            
   \bibliography{bib/literature}
   
      %  \DefineBibliographyStrings{ngerman}{%
         %   bibliography     = {Literaturverzeichnis},  % = \bibname
         %   references       = {Literatur},             % = \refname
     %   }
        \defbibnote{alphabetic}{%
            Die Literaturangaben sind alphabetisch nach den Namen
            der Autoren sortiert. Bei mehreren Autoren wird nach
            dem ersten Autor sortiert.\par
            Und mit dem neuen \LPack{biblatex}-Paket funktioniert
            das auch, wie man unschwer erkennen kann.\par\bigskip
        }


% -------------------------------------------------------------------------------------------------
%% declare author names as "last, first".
%% Either for the first author only or for all authors
%\DeclareNameFormat{author}{%
%    \ifthenelse{\value{listcount}=1}
%        {#1%                                            % first author
%            \ifblank{#3}{}{\addcomma\space #3}}
%        {#1%                                            % all the other authors (last, first)
%            \ifblank{#3}{}{\addcomma\space #3}}%
%%        {\ifblank{#3}{}{#3\space}%                      % all the other authors (first last)
%%            #1}%
%    \ifthenelse{\value{listcount}<\value{liststop}}
%        {,\space}
%            {}
%}
%
%%http://projekte.dante.de/DanteFAQ/BiblatexReihenfolgeAutoren
%\DeclareNameFormat{last-first}{%
%  \iffirstinits
%    {\usebibmacro{name:last-first}{#1}{#4}{#5}{#7}}
%    {\usebibmacro{name:last-first}{#1}{#3}{#5}{#7}}%
%  \usebibmacro{name:andothers}}
%\DeclareNameFormat{labelname}{%
%   \ifuseprefix
%     {\usebibmacro{name:last-first}{#1}{#4}{#5}{#8}}
%     {\usebibmacro{name:last-first}{#1}{#4}{#6}{#8}}%
%   \usebibmacro{name:andothers}}

%  \DefineBibliographyStrings{english}{%
%    typeeditor = {{}{}},
%    typeeditors = {{}{}},
%    in = {{}{}},
%    inseries = {{}{}},
%    byeditor = {{}{}}
%  }

%-------------------------------------------------------------------------------------------------

%% http://www.golatex.de/biblatex-anpassen-die-x-te-frage-t4657.html
\renewbibmacro*{journal+issuetitle}{%
  \usebibmacro{journal}%
  \setunit*{\addcomma\space}%
  \iffieldundef{series}
    {}
    {\newunit
     \printfield{series}%
     \setunit{\addcomma\space}}%
  \printfield{volume}%
  \setunit*{\addcomma\space}%
  \printfield{number}%
  \setunit{\addcomma\space}%
  \printfield{eid}%
  \setunit{\addspace}%
  \usebibmacro{issue+date}%
  \setunit{\addcolon\space}%
  \usebibmacro{issue}%
  \newunit}

\DeclareFieldFormat[article]{volume}{\bibstring{jourvol}~#1}
\DeclareFieldFormat[article]{number}{\bibstring{number}~#1} 
%\DeclareFieldFormat[article]{edition}{\bibstring{edition}~#1} 

%% http://mrunix.de/forums/showthread.php?t=67386
\DefineBibliographyStrings{english}{jourvol={Vol\adddot}} 
\DefineBibliographyStrings{english}{number={No\adddot}} 
\DefineBibliographyStrings{english}{edition={Ed\adddot}} 

\AtBeginBibliography{%
  % Setzt die Autoren-Vornamen auf Kapit�lchen 
  \renewcommand*{\mkbibnamefirst}{\textsc}
  \renewcommand*{\mkbibnamelast}{\textsc}
  \renewcommand*{\mkbibnameprefix}{\textsc}
  \renewcommand*{\mkbibnameaffix}{\textsc}
  
  %%Doppelpunkt nach Namen, kein Punkt
  %\renewcommand*{\labelnamepunct}{\addcolon\space} 
  
  \DeclareFieldFormat{name}{\textsc{#1\isdot}}
  \DeclareFieldFormat{title}{\mkbibemph{#1\isdot}}
  
  %%\DeclareFieldFormat[article]{title}{#1}
  %%\DeclareFieldFormat[article]{title}{\mkbibquote{#1}}
  \renewcommand*{\mkbibquote}[1]{\mkbibemph{#1\isdot}}
 
  % http://tex.stackexchange.com/ ...
  % questions/16716/spell-out-volume-and-edition-in-words-biblatex-in-german
  %\renewcommand*{\mkbibordedition}[1]{\Ordinalstringnum{#1}[f]}
  \renewcommand*{\mkbibordedition}[1]{\ordinalnum{#1}}
}