% ~~~~~~~~~~~~~~~~~~~~~~~~~~~~~~~~~~~~~~~~~~~~~~~~~~~~~~~~~~~~~~~~~~~~~~~~
% Math Packages
% ~~~~~~~~~~~~~~~~~~~~~~~~~~~~~~~~~~~~~~~~~~~~~~~~~~~~~~~~~~~~~~~~~~~~~~~~
 
% *** Mathematik **************************************
%
% amsmath schon vorher geladen da es vor pst-pdf geladen werden muss
 
 
%%% Doc: ftp://tug.ctan.org/pub/tex-archive/macros/latex/contrib/mh/doc/mathtools.pdf
% Erweitert amsmath und behebt einige Bugs
\usepackage[fixamsmath,disallowspaces]{mathtools}

 
%%% Doc: http://www.ctan.org/info?id=fixmath
% LaTeX's default style of typesetting mathematics does not comply
% with the International Standards ISO31-0:1992 to ISO31-13:1992
% which indicate that uppercase Greek letters always be typset
% upright, as opposed to italic (even though they usually
% represent variables) and allow for typsetting of variables in a
% boldface italic style (even though the required fonts are
% available). This package ensures that uppercase Greek be typeset
% in italic style, that upright $\Delta$ and $\Omega$ symbols are
% available through the commands \upDelta and \upOmega; and
% provides a new math alphabet \mathbold for boldface
% italic letters, including Greek.
% \usepackage{fixmath}
 
%%% Doc: ftp://tug.ctan.org/pub/tex-archive/macros/latex/contrib/onlyamsmath/onlyamsmath.dvi
% Warnt bei Benutzung von Befehlen die mit amsmath inkompatibel sind.
%\usepackage[
  %all,
  %warning
%]{onlyamsmath}
 
 
%------------------------------------------------------

 
%%% Doc: ftp://tug.ctan.org/pub/tex-archive/macros/latex/contrib/misc/braket.sty
\usepackage{braket}  % Quantenmechanik Bracket Schreibweise
 
%%% Doc: ftp://tug.ctan.org/pub/tex-archive/macros/latex/contrib/misc/cancel.sty
%\usepackage{cancel}  % Durchstreichen
 
%%% Doc: ftp://tug.ctan.org/pub/tex-archive/macros/latex/contrib/mh/doc/empheq.pdf
%\usepackage{empheq}  % Hervorheben
 
%%% Doc: ftp://tug.ctan.org/pub/tex-archive/info/math/voss/mathmode/Mathmode.pdf
%\usepackage{exscale} % Skaliert Mathe-Modus Ausgaben in allen Umgebungen richtig.
 
%%% Doc: ftp://tug.ctan.org/pub/tex-archive/macros/latex/contrib/was/icomma.dtx
% Erlaubt die Benutzung von Kommas im Mathematikmodus
\usepackage{icomma}

%%% Doc: http://www.ctex.org/documents/packages/special/units.pdf
% \usepackage[nice]{nicefrac}

%%% Doc: ftp://tug.ctan.org/pub/tex-archive/macros/latex/contrib/numprint/numprint.pdf
% Modify printing of numbers
%\usepackage{numprint}

%------------------------------------------------------

% For nice arrows over vectors
% \usepackage{esvect}

% Using bold greek letters with \bf
\usepackage{bm}

% Allows arrows in matrices: ldelim, rdelim
\usepackage{bigdelim}

\usepackage[amsmath,hyperref,framed]{ntheorem} %,thmmarks <--endmark
  \theoremstyle{plain}
    %% Update \crefname for cleveref
     \newtheorem{theorem}{Theorem}[section]
%    \newtheorem{lemma}[theorem]{Lemma}
%    \newtheorem{proposition}[theorem]{Proposition}
%    \newtheorem{corollary}[theorem]{Corollary}
%    \newtheorem{remark}[theorem]{Remark}
%     \newtheorem{definition}[theorem]{Definition}
%    \newtheorem{example}[theorem]{Example}
%    \newtheorem{conjecture}[theorem]{Conjecture}
  
  %Proof environment
  \theoremstyle{nonumberplain}
    \theoremseparator{.}
    \theoremheaderfont{\bfseries}
    \theorembodyfont{\normalfont}
    \theoremsymbol{$\blacksquare$}
      \RequirePackage{amssymb}
    \newtheorem{proof}{Proof}
    
  %Fix to Definition as math is not in bold in "`Def. 1 ($s$-step)"'
      \makeatletter
    \let\copy@theorem@headerfont=\theorem@headerfont
    \newcommand{\my@theorem@headerfont}{%
        \boldmath\copy@theorem@headerfont\unboldmath
      }
    \let\theorem@headerfont=\my@theorem@headerfont
      \makeatother
   
%\usepackage{framed}
%\usepackage{thmbox}
\theoremstyle{nonumberplain}
\theoremseparator{:}
\setlength\theorempreskipamount{0.8cm}\setlength\theorempostskipamount{0.8cm}

\newtheorem{definition}[theorem]{Definition}



\usepackage{siunitx}
    \sisetup{
        %color=UnitColor,       %% Pr�fen, ob alle Einheiten mit siunitx erstellt wurden; vor Druck auskommentieren
        %forbid-literal-units,
        input-signs={+-<>\leq\geq\pm\mp\approx\sim},
        %per-mode=symbol-or-fraction,
        %retain-explicit-plus,
        %
        load-configurations={
            abbreviations,
        },
        %
				group-separator={,},
				detect-all,
				round-mode=places,round-precision=4,
				exponent-product=\cdot,
        retain-zero-exponent=true,
    }
    %\addto\extrasngerman{\sisetup{locale = DE}}

% Defines \Ordinalstringnum to convert number "2" to literal expression "second"
\usepackage{fmtcount}

% Allows page breaks in mathemaical environments
\allowdisplaybreaks

%------------------------------------------------------

 
%%% Tauschen von Epsilon und andere:
% \let\ORGvarrho=\varrho
% \let\varrho=\rho
% \let\rho=\ORGvarrho
%
\let\ORGvarepsilon=\varepsilon
\let\varepsilon=\epsilon
\let\epsilon=\ORGvarepsilon
%
% \let\ORGvartheta=\vartheta
% \let\vartheta=\theta
% \let\theta=\ORGvartheta
%
% \let\ORGvarphi=\varphi
% \let\varphi=\phi
% \let\phi=\ORGvarphi