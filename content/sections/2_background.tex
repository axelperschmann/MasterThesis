\chapter{Background}
\label{chap:background}
This chapter gives a general introduction into the vocabulary of financial computing and the machine learning techniques employed.

\section{Exchange Markets, Bitcoins and Trading Basics}
An exchange is a market, where financial instruments are sold and bought. It is typically organized by a broker, which can be both, an individual or a firm, executing buy and sell orders on behalf of dealers for a certain fee or commission. The respective prices are determined by the current market situations, in particular by supply and demand.\\

Specialized exchanges concentrate on certain sub-types of financial instruments and offer a trading venue for those willing to buy and sell these instruments. Some of them are listed below:
 
\begin{description}
\item[Stock Exchange Market] A stock exchange or bourse provides companies access to investment capital in exchange to a share of ownership. Especially in times with notoriously low interest rates, investors tend to accept the greater risk of business development over a risk free, but faint investment, to grow their assets.\\
\Eg NASDAQ, Deutsche B�rse, \dots{}
\item[Commodity exchange market] Commodity exchange markets allow for speculations with goods like oil, gold, corn, \dots{} \\
\Eg Eurex, \dots{}
\item[Foreign exchange market] Foreign exchange (short: forex) is considered the largest financial market in the world. The forex market is responsible for determining currency exchange rates.
\item[Bitcoin exchange market] x\\
\Eg Poloniex, \dots{}
\end{description}

Most modern markets are usually fully electronic{\color{red} .......}

\subsection{Ask and Bid}
Most exchange markets function after the so called auction market model, where the exchange acts as a mediator between buyers and sellers to ensure fair trading. Here buyers can \emph{bid} a price they are willing to pay for a certain number of shares and sellers can \emph{ask} a price they are aiming to make with a number of shares.
The highest of all bids is called the \emph{bid price}, the lowest of all offers ist called the \emph{ask price}. Together they represent the current price at which an instrument is traded.\\

\subsection{Limit Order Book and Market Depth}
A limit orderbook reflects supply (asks) and demand (bids) for a particular financial instrument. It is usually maintained by the trading venue and lists the number of shares being bid or offered, organized by price levels in two opposing books. Incoming orders are constantly appended to this highly dynamic list, while a matching engine cautiously resolves any inconsistencies (\ie overlaps) between asks and bids by mediating between the involved parties.\\

It is usually not before the matching engine has arranged an actual trade, that a trading venue claims a certain percentage of the turnover as a service fee. To encourage active market participation, the pure submission, revision and cancelation of orders is typically free of charge.

\begin{table}
\centering
\begin{tabular}{lrlrrr}
\toprule
{} &  Amount &    Type &  Volume &  VolumeAcc &  norm\_Price \\
\midrule
31.00 &   \color{mymauve}200.0 &     ask &  6200.0 &     8425.0 &    1.074533 \\
30.00 &    \color{mygreen}50.0 &     ask &  1500.0 &     2225.0 &    1.039871 \\
29.00 &    \color{red}25.0 &     ask &   725.0 &      725.0 &    1.005208 \\
28.85 &     NaN &  center &     NaN &        NaN &         NaN \\
28.70 &   200.0 &     bid &  5740.0 &     5740.0 &    0.994810 \\
28.50 &   100.0 &     bid &  2850.0 &     8590.0 &    0.987877 \\
28.00 &   300.0 &     bid &  8400.0 &    16990.0 &    0.970546 \\
\bottomrule
\end{tabular}
\caption[Exemplary snapshot of a limit orderbook]{Exemplary snapshot of a limit orderbook for stocks of AIWC\protect\footnotemark}
\label{table:orderbook:example}
\end{table}
\footnotetext{Acme Internet Widget Company}

\Cref{table:orderbook:example} shows a limit orderbook snapshot up to a market depth of 3, as seen by market participants. Here Alice offers {\color{red}25} shares per 29\$, Bob and Cedar offer {\color{mygreen}20 and 30} shares respectively per 30\$ and David offers {\color{mymauve}200} shares per 31\$.\\


\label{sec:marketdata:levels}
Based on their trading needs, traders can typically choose between multiple levels of real-time market data.
\begin{description}
\item[Level 1 Market Data] Basic informations only:\\
Bid price + size, Ask price + size, Last price + size
\item[Level 2 Market Data] Additional access to the orderbook.\\
Usually data providers display the orderbook only up to a certain market depth $m$, \ie the lowest $m$ asks and the highest $m$ bids.
\item[Level 3 Market Data] Full data access.\\
Typically only accessible for the market maker.

\end{description}

\subsection{Slippage}
Slippage is defined as the difference between expected and achieved price at which a trade is executed. Slippage may occur due to delayed trade execution. Especially during periods of high volatility, markets might change faster than the order takes to be executed. Slippage is also liked to the order size, as larger orders tend to \emph{eat} into the opposing book and are fulfilled at successively worse price levels.\\
Slippage can be both positive or negative, depending on the current market movements and must be taken into account by serious investors.


\subsection{Order Types}
\label{chap:ordertypes}
Investors can execute orders of different types, of which the most common ones are described below:

\begin{description}
\item[Market Orders] are the most simple form of orders. Here, the investor only specifies the number of shares he want's to buy/sell and the full order is executed immediately, at any price. Especially for large-scale traders or traders with level 1 data access only, these simple market orders are rather hazardous, since the achieved price can significantly differ from the expected price due to sparse supply and demand.

\item[Limit Orders] additionally feature a worst price, \ie the highest price a buyer is willing to pay per share or, respectively the lowest price a seller is willing to make per share. Limit orders are immediately placed into the orderbook and (partially) executed, once the matching engine finds a corresponding trade in the opposing book.\\
Limit orders reduce the risk of slippage, but do not guarantee execution.

\item[Hidden Orders] are placed into the market makers internal orderbook, but not displayed to other market participants with level 2 market data access. They represent a simple solution to large-scale investors seeking anonymity in the market, aiming to obfuscate their trading intention from other market participants.

\end{description}

\subsection{Trading strategies}
\label{chap:tradingstrategies}
An order placement typically originates from a carefully considered \emph{trading strategy}. An \emph{active} trading strategy buys and sells instruments frequently based on short-term price movements, whereas a \emph{passive} trading strategy such as \emph{Buy-And-Hold} believes in long-term price movements eventually outweighting any short-term fluctuations.\\

As the execution of trades typically implies trading costs and slippage, these have to be taken into account. Particularly active traders with a high order quantity and large-scale investors with high order volumes are concerned with this burden. The order type chosen has a major impact on speed of execution and slippage generated.\\

While \emph{limit orders} reduce the risk of slippage, they do not guarantee full order execution. This leads to the important problem of \emph{optimized trade execution}, which frequently occurs in the domain of financial computing. In its simplest form, the problem is defined by a particular financial instrument (here: Bitcoins), which must be bought or sold within a fixed time horizon, for the best achievable share price.\\

In \cite{Nevmyvaka2005SubmitAndLeave} Nevmyvaka \etal introduce a \emph{Submit \& Leave} strategy, which cleverly combines market and limit orders: After an initial limit order submission, the order is left on the market for a predefined time horizon, after which it's unexecuted part is transformed into a market order and thus executed completely. They later extended their strategy to a \emph{Submit \& Revise} strategy \cite{Nevmyvaka:2006}, where the order limit may be revised at discrete time steps, depending on trade progress and market changes.


\section{Bitcoins}
\label{chap:bitcoins}

\begin{figure}[ht]
	\centering
	[placeholder]
	\caption{bitcoin volatility}
	High frequency tick data, compared to minutely snapshots.
	
	\label{fig:bitcoinvolatility}
\end{figure}

\subsection{Marktreaktionen}
"Bitcoin ist eine W�hrung, die �u�erst sensibel auf Nachrichten reagiert. Begr�ndet wird dies vor allem durch die M�glichkeit, st�ndig am Markt teilnehmen zu k�nnen: Es gibt keine zentrale Ausgabestelle mit geregelten Handelszeiten, an die man gebunden ist."

"Auch die Tatsache, dass viele Anf�nger in Bitcoins investieren, f�hrte bereits in der Vergangenheit zu den ein oder anderen Panikverk�ufen. Wer sein Geld in Bitcoins investieren m�chte, kann die meist lukrative M�glichkeit nutzen, sollte sich jedoch regelm��ig �ber Marktver�nderungen informieren.

Da viele Investoren schnell auf Meldungen reagieren, kann es innerhalb von Stunden zu gro�en Kursverlusten oder Gewinnen kommen."
\url{https://www.btc-echo.de/bitcoin-trading-tipps-prinzipien-des-bicoin-handels_2015022502/}


\section{Supervised Learning}
Supervised learning is a subdomain of machine learning, where a function is learned from labeled training data $\{ (x_1, y_1), ..., (x_N, y_N) \} $. Each training sample maps a feature vectors $x_i \in X$ to a desired target value or label $y_i \in Y$. Target values may either be categorial, making the learning task a \emph{classification} problem, or continuous, making the learning task a \emph{regression} problem.

A supervised learning algorithms seeks to find a general function $g_w()$ (or it's parameters $w$), such that $g_w(x_i) \approx y_i | i \leq N$. The learned function should ideally avoid overfitting by finding a generalization to previously unseen data.


Markow Decision Process

Value Function and Bellmann Equation

Value Iteration

Q-Learning

\subsection{Logistic Regression}
bla

\subsection{Random Forest}
bla



\section{Reinforcement Learning}
Reinforcement learning is a subdomain of machine learning, where strategies are learned by an \emph{agent} interacting with its environment. Rather than learning from labeled training data, the agent applies a \emph{trial and error} pattern and exploits external rewards to find actions, maximizing his expected future reward.

\subsection{Dynamic backward programming}
bla

\subsection{Tree-Based Batch Mode Reinforcement Learning}
bla

\subsection{Neural Fitted Q-learning}
bla


\cleardoublepage{}






