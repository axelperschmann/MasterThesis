\chapter{Experiments}
\label{chap:experiments}
This chapter is about a concrete, real world example.



\begin{itemize}
\item actions represent deviation from ask, not bid
\item Forward Sampling
\begin{itemize}
\item Growing batch learning
\item Experience Replay (NN\_Agent)
\item Exploration vs avoidance of repeatedly trying same actions.
\end{itemize}
\begin{itemize}
\item Markov Property violated
\item Realistic samples, no rounding. Better fit for \emph{curious} masterbook shapes as described in \Cref{chap:modelcorrectness}?!
\end{itemize}
\item Function Approximation
\begin{itemize}
\item RandomForest (BatchTree Agent)
\item NN Agent
\end{itemize}
\item Different Market Variables:
\begin{itemize}
\item Immediate Slippage/MarketPrice Imbalance
\item MarketPrice Spread (buy vs. sell)
\end{itemize}
\item Cost function
\begin{itemize}
\item Slippage based on initial\_center
\item Improvement over MarketPrice??
\end{itemize}
\end{itemize}


\section{RL Agents}
bla

\subsection{QTable Agent}
bla

\subsection{BatchTree Agent}
bla

\subsection{NN Agent}
bla


\section{Action-Limit Mapping}
\label{chap:exp:actionlimitmapping}
Throughout the rest of this thesis, the same 15 actions are evaluated: $[-4, -3, ..., 8, 9, 10]$, corresponding to limit factors of $[0.996, 0.997, ..., 1.008, 1.009, 1.01]$\\

\begin{figure}[ht]
	\centering
	[placeholder]
	\caption{Comparison of action-limit map functions.}
	\label{fig:actionlimitmapping}
\end{figure}

\section{Additional Market Variables}
\label{chap:exp:additionalmarketvars}


\section{Backward approach}

\subsection{QTable Agent + additional orderbook features}
bla

\subsection{QTable Agent + additional technical indicators}

\subsection{QTable Agent vs. BatchTree Agent vs. NN Agent}
bla




\section{Forward approach}
\label{chap:forwardlearning}
bla

\subsection{Motivation}
Markov Assumption is wrong: States do depend on path chosen before.

Data efficency. Learning with fewer samples

No need for discretization of volume.

\subsection{Results}
bla





\cleardoublepage{}