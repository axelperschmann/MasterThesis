\KOMAoptions{titlepage=false}
\renewcommand*{\abstractname}{\Large\rmfamily\bfseries \color{black} Abstract}
\begin{abstract}
\normalfont\normalsize
This thesis tackles the important problem of optimized trade execution, which aims to obtain the best possible price for a trade instructed by higher level investment decisions. In its simplest form the problem is defined by a particular financial instrument, which must be bought or sold within a fixed time horizon, while minimizing the expenditure for doing so.

Modifications to an existing reinforcement learning approach are made and a novel forward-learning algorithm is proposed, outperforming the cost avoiding capabilites of all previous methods.

\end{abstract}\vspace{5cm}

\renewcommand*{\abstractname}{\Large\rmfamily\bfseries \color{black} Zusammenfassung}
\begin{abstract}
\normalfont\normalsize
Diese Thesis befasst sich mit dem Problem der Optimalen Handelsausf�hrung, welches darin besteht den bestm�glichen Preis f�r ein Finanzprodukt zu erzielen, dessen Handel von einer h�heren Investitionsstrategie angeordnet wurde. Am leichtesten l�sst sich dieses Problem beschreiben durch ein bestimmtes Finanzprodukt, welches innerhalb eines festgelegten Zeitraumes zum bestm�glichen Preis gekauft oder verkauft werden soll.

Hierf�r wurde ein bestehender Reinforcement Learning Ansatz verbessert und ein neuartiger, vorw�rts lernender Algorithm vorgestellt, der die kostenreduzierenden F�higkeiten der vorherigen Methoden �berbietet. 

\end{abstract}

\clearpage{}